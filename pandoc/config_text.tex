%  Cofiguração LaTeX para gerar PDF com o Pandoc.


% --- Configurações de Layout e Margens ---
\usepackage[margin=1.5cm]{geometry} % Margens: Ajuste para o valor desejado (ex: 2cm, ou left=2cm,right=2cm,top=2.5cm,bottom=2.5cm)
\usepackage[fontsize=14pt]{scrextend}  % Altera o tamanho da fonte

% \usepackage{helvet} % Para a fonte Helvetica
\renewcommand{\familydefault}{\sfdefault} % Define a família de fontes padrão como sans-serif (Helvetica)


% --- Configurações de Fonte (Requer XeLaTeX ou LuaLaTeX instalado, ou comente o {fontspec} e desconcidere o uso.) ---
%
% Adicione \'--pdf-engine=xelatex\' ou \'--pdf-engine=lualatex\' ao comando. 
% Exemplos:
%           pandoc <arquivo.md> -o <arquivo_gerado.pdf> --toc -H config_text.tex --pdf-engine=xelatex
%


 \usepackage{fontspec} % Pacote para usar fontes do sistema
 \setmainfont{Liberation Sans} % Define a fonte principal do documento
% Opcional: Para fontes monoespaçadas (usadas em blocos de código e `texttt`)
% Escolha uma fonte monoespaçada que você tenha instalada, como Consolas, Fira Code, Hack, Courier New, etc.
 \setmonofont{Fira Code} 
\usepackage{textcomp} % para melhorar a exibição de operadores e outros símbolos 

% --- Configurações de Cores e Destaque ---
\usepackage{xcolor} % Pacote para cores

% Definição de cores personalizadas (opcional, mas recomendado para consistência)
\definecolor{comando_cor}{RGB}{0,0,139} % Azul escuro
\definecolor{atalho_cor}{RGB}{0,100,0}  % Verde escuro
\definecolor{script_cor}{RGB}{205,133,63} % Laranja/Marrom claro

% Comandos personalizados para destaque no Markdown (usar como raw LaTeX)
% Ex: \comando{Ctrl + C}
\newcommand{\comando}[1]{\textbf{\textcolor{comando_cor}{#1}}}
\newcommand{\atalho}[1]{\textbf{\textcolor{atalho_cor}{#1}}}
\newcommand{\script}[1]{\texttt{\textcolor{script_cor}{#1}}} % \texttt para texto monoespaçado

% --- Outras configurações úteis (opcional) ---
% Para ajustar o espaçamento entre linhas (se necessário)
 \linespread{1.1} % Ajusta o espaçamento para 110% do padrão


% --- CONFIGURAÇÕES PARA TABELAS ---
\usepackage{booktabs} % Para tabelas com linhas profissionais (toprule, midrule, bottomrule)
\usepackage{array}    % Para maior controle sobre colunas de tabela
\usepackage{longtable} % Para tabelas que podem quebrar entre páginas

% Ajustes de espaçamento e alinhamento (opcional)
\setlength{\tabcolsep}{8pt} % Espaçamento entre colunas
\renewcommand{\arraystretch}{1.2} % Espaçamento entre linhas

% Cores para as linhas da tabela (opcional, requer xcolor)
% Já temos o pacote xcolor carregado, então podemos usar.
\definecolor{lightgray}{rgb}{0.9,0.9,0.9} % Cor cinza claro

% Comando para sombrear linhas alternadas (opcional)
\usepackage{colortbl}
\rowcolors{2}{white}{lightgray} % Começa na linha 2, alterna branco e cinza claro


% Carrega o pacote para renderizar código
\usepackage{listings}

% Definição de cores para o realce de sintaxe
\definecolor{codegreen}{rgb}{0.0,0.6,0.0}
\definecolor{codegray}{rgb}{0.5,0.5,0.5}
\definecolor{codepurple}{rgb}{0.58,0.0,0.82}
\definecolor{backcolour}{rgb}{0.98,0.98,0.98}

% Configurações globais para todos os blocos de código com realce de sintaxe
\lstset{ 
  breaklines=true,            % Quebrar linhas longas automaticamente
  postbreak=\mbox{\textcolor{red}{$\hookrightarrow$}\space}, % Símbolo para indicar quebra de linha
  numbers=left,               % Números de linha à esquerda
  numberstyle=\tiny\color{gray}, % Estilo dos números de linha
  frame=single,               % Adiciona uma borda ao redor do código
  captionpos=b,               % Posição da legenda (b=bottom)
  backgroundcolor=\color{backcolour},
  commentstyle=\color{codegreen},
  keywordstyle=\color{magenta},
  numberstyle=\tiny\color{codegray},
  stringstyle=\color{codepurple},
  basicstyle=\ttfamily\footnotesize,
  breakatwhitespace=false,
  breaklines=true,
  captionpos=b,
  keepspaces=true,
  numbersep=5pt,
  showspaces=false,
  showstringspaces=false,
  showtabs=false,
  tabsize=2,
}

