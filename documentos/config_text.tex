
%  Cofiguração LaTeX para gerar PDF com o Pandoc.


% --- Configurações de Layout e Margens ---
\usepackage[margin=1.5cm]{geometry} % Margens: Ajuste para o valor desejado (ex: 2cm, ou left=2cm,right=2cm,top=2.5cm,bottom=2.5cm)
\usepackage[fontsize=14pt]{scrextend}  % Altera o tamanho da fonte

% \usepackage{helvet} % Para a fonte Helvetica
\renewcommand{\familydefault}{\sfdefault} % Define a família de fontes padrão como sans-serif (Helvetica)


% --- Configurações de Fonte (Requer XeLaTeX ou LuaLaTeX instalado) ---

 \usepackage{fontspec} % Pacote para usar fontes do sistema
 \setmainfont{Liberation Sans} % Define a fonte principal do documento como Arial
% Opcional: Para fontes monoespaçadas (usadas em blocos de código e `texttt`)
% Escolha uma fonte monoespaçada que você tenha instalada, como Consolas, Fira Code, Hack, Courier New, etc.
 \setmonofont{Fira Code} 

% --- Configurações de Cores e Destaque ---
\usepackage{xcolor} % Pacote para cores

% Definição de cores personalizadas (opcional, mas recomendado para consistência)
\definecolor{comando_cor}{RGB}{0,0,139} % Azul escuro
\definecolor{atalho_cor}{RGB}{0,100,0}  % Verde escuro
\definecolor{script_cor}{RGB}{205,133,63} % Laranja/Marrom claro

% Comandos personalizados para destaque no Markdown (usar como raw LaTeX)
% Ex: \comando{Ctrl + C}
\newcommand{\comando}[1]{\textbf{\textcolor{comando_cor}{#1}}}
\newcommand{\atalho}[1]{\textbf{\textcolor{atalho_cor}{#1}}}
\newcommand{\script}[1]{\texttt{\textcolor{script_cor}{#1}}} % \texttt para texto monoespaçado

% --- Outras configurações úteis (opcional) ---
% Para ajustar o espaçamento entre linhas (se necessário)
 \linespread{1.1} % Ajusta o espaçamento para 110% do padrão

